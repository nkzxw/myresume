% !TEX TS-program = xelatex
% !TEX encoding = UTF-8 Unicode
% !Mode:: "TeX:UTF-8"

\documentclass[10pt]{article} % Default font size
\usepackage[a4paper, hmargin=25mm, vmargin=30mm, top=20mm]{geometry} % Use A4 paper and set margins
\usepackage{fancyhdr} % Customize the header and footer
%\usepackage{lastpage} % Required for calculating the number of pages in the document
%\usepackage{hyperref} % Colors for links, text and headings
\setcounter{secnumdepth}{0} % Suppress section numbering
%\usepackage[proportional,scaled=1.064]{erewhon} % Use the Erewhon font
%\usepackage[erewhon,vvarbb,bigdelims]{newtxmath} % Use the Erewhon font
\usepackage[utf8]{inputenc} % Required for inputting international characters
\usepackage[T1]{fontenc} % Output font encoding for international characters
\usepackage{fontspec} % Required for specification of custom fonts
\usepackage[BoldFont,SlantFont,CJKchecksingle]{xeCJK}

\setmainfont[
Path = ./fonts/,
Extension = .otf,
BoldFont = Erewhon-Bold,
ItalicFont = Erewhon-Italic,
BoldItalicFont = Erewhon-BoldItalic,
SmallCapsFeatures = {Letters = SmallCaps}
]{Erewhon-Regular}

\setCJKmainfont[
Path=./fonts/,
Extension=.TTF,
BoldFont=FZLanTingHeiS-B-GB,
]{FZLanTingHeiS-R-GB}

\usepackage{color} % Required for custom colors
\definecolor{slateblue}{rgb}{0.17,0.22,0.34}

\usepackage{sectsty} % Allows customization of titles
\sectionfont{\color{slateblue}} % Color section titles

%\fancypagestyle{plain}{\fancyhf{}\cfoot{\thepage\ of \pageref{LastPage}}} % Define a custom page style
\pagestyle{plain} % Use the custom page style through the document
\renewcommand{\headrulewidth}{0pt} % Disable the default header rule
\renewcommand{\footrulewidth}{0pt} % Disable the default footer rule

\setlength\parindent{0pt} % Stop paragraph indentation

% Non-indenting itemize
\newenvironment{itemize-noindent}
{\setlength{\leftmargini}{0em}\begin{itemize}}
{\end{itemize}}

% Text width for tabbing environments
\newlength{\smallertextwidth}
\setlength{\smallertextwidth}{\textwidth}
\addtolength{\smallertextwidth}{-2cm}

\newcommand{\sqbullet}{~\vrule height 1ex width .8ex depth -.2ex} % Custom square bullet point definition

\renewcommand{\title}[1]{
{\huge{\color{slateblue}\textbf{#1}}}\\ % Header section name and color
\rule{\textwidth}{0.5mm}\\ % Rule under the header
}

\newcommand{\education}[1]{
\begin{tabbing}
\hspace{2cm} \= \hspace{1cm} \= \kill
#1
\end{tabbing}
}

\newcommand{\job}[4]{
\begin{tabbing}
\hspace{2cm} \= \kill
\textbf{#1} \> {#2} \\
\>\+ \textit{#3} \\
\begin{minipage}{\smallertextwidth}
\vspace{2mm}
#4
\end{minipage}
\end{tabbing}
\vspace{2mm}
}

\newcommand{\skillgroup}[2]{
\begin{tabbing}
\hspace{5mm} \= \kill
\sqbullet \>\+ \textbf{#1} \\
\begin{minipage}{\smallertextwidth}
\vspace{2mm}
#2
\end{minipage}
\end{tabbing}
} % Include the file specifying document layout
\begin{document}

\title{张兴伟}
\parbox{2pt}{
\begin{tabbing}
\hspace{1.5cm} \= \hspace{4cm} \kill % Spacing within the block
\textbf{地址:} \>北京\\
\textbf{年龄:} \>29
\end{tabbing}
}
\hfill % Horizontal space between the two blocks
\parbox{2pt}{
\begin{tabbing}
\hspace{1.5cm} \= \hspace{4cm} \kill % Spacing within the block
\textbf{手机号:} \>(+86)123-123-12312\\ 
\textbf{邮箱:} \>zxw5450@foxmail.com
\end{tabbing}
}

\section{个人总结}
毕业后一直从事文件系统存储相关工作,工作上认真负责,自我驱动力强。\\对文件系统相关的底层技术有所研究,对大数据开发,分布式存储,高性能应用感兴趣。
\section{教育背景}
\education{
\textbf{2012-2014} \> 南开大学 \>硕士\+\\[5pt]%设置5pt换行间隔
\textit{专业:计算机技术}
}
\education{
\textbf{2008-2012} \> 南开大学 \>本科\+\\[5pt]%设置5pt换行间隔
\textit{专业:计算机科学与技术}
}
\section{工作经历}
%\job
%{Sep 2011 -}{Present}
%{Lehman Brothers, 1234 Mario Park, San Francisco, CA, United States}
%{http://www.lehmanbrothers.com}
%{Senior Developer / Technical Team Lead}
%{Lorem ipsum dolor sit amet, consectetur adipiscing elit. Duis elementum nec dolor sed sagittis. Cras justo lorem, volutpat mattis lacus vel, consequat aliquam quam. Interdum et malesuada fames ac ante ipsum primis in faucibus.\\
%\rule{0mm}{5mm}\textbf{Technologies:} Ruby on Rails 2.3, Amazon EC2, NoSQL data stores, memcached, collaborative matching, Facebook Graph API.}

\job
{2014-2018}
{北京中科蓝鲸信息技术有限公司}
{高级研发工程师}
{主要负责Bwfs分布式文件系统的功能开发和实现功能稳定。
\begin{itemize-noindent}
\item{对自研的分布式文件系统进行bug定位和功能改进}
\item{技术:多线程、windows文件系统驱动、ddk、wdk}
\item{实现Windows内核态rpc}
\item{技术:wsk、tdi socket、双向rpc}
\item{实现文件系统跨平台的文件软连接、硬链接}
\item{技术:文件系统底层知识}
\item{实现分布式文件系统的delegation、session、分布式锁}
\item{技术:Linux平台C/C++服务开发、zeromq、fcn、pnfs、消息队列}
\item{文件系统服务HA高可用功能开发}
\item{技术:浮动ip,心跳检测,rpc重连,状态重建}
\end{itemize-noindent}
}
\section{技术能力}
\skillgroup{编程语言}
{
\textit{精通C/C++} 、\textit{熟悉Java} 、\textit{熟悉Python}
}
\skillgroup{分布式文件系统}
{
\textit{Linux平台下文件系统相关的服务开发}
}
\skillgroup{Windows驱动开发}
{
\textit{文件系统驱动,过滤驱动开发,性能调优}
}
\end{document}